%%%%%%%%%%%%%%%%%%%%%%%%%%%%%%%%%%%%%%%%%%%%%%%%%%%%%%%%%%%%%%%%%%%%%%%%%%%%%%%
% intro.tex: Introduction to the thesis
%%%%%%%%%%%%%%%%%%%%%%%%%%%%%%%%%%%%%%%%%%%%%%%%%%%%%%%%%%%%%%%%%%%%%%%%%%%%%%%%
\chapter{Introduction}
\label{intro_chapter}
%%%%%%%%%%%%%%%%%%%%%%%%%%%%%%%%%%%%%%%%%%%%%%%%%%%%%%%%%%%%%%%%%%%%%%%%%%%%%%%%
Particle physics is the study of the particles that make up the universe, as well as the forces that affect them. The most widely accepted model is known as the Standard Model, and has been shown to accurately explain and predict many processes. However, despite this, the Standard Model is limited. For instance, it does not account for dark matter, dark energy or even gravity. A limitation that is of particular importance to hadron experiments is with respect to the strong force, since it is very difficult to use that component of the Standard Model to make many types of predictions. For this reason many models do not use the Standard Model for predicting certain effects of the strong force. 

This thesis attempts to test the effects of modifying the ``hadronizer" which simulates the effects of low energy strong interactions. These effects are probed by measuring the leptons that result from a \Z decay. This was done due to both the \Z and its decay products lack of interaction with the strong force.  Therefore once the Z is produced the strong force will have no other effects on the measured values. Thus, by measuring transverse momentum of the \Z(\bosonpt) we can infer information about the transverse momentum of the particles that produced it. In place of measuring the \bosonpt of the Z directly the novel variable \phistar is used. This variable was chosen due to being highly correlated with \bosonpt  while having a smaller percentage uncertainty.  

The data used was collected using the Compact Muon Solenoid detector(CMS) at the Large Hadron Collider(LHC) in 2012. The data used  contains $\SI{19.7}{fb}^{-1}$ of integrated luminosity at a center of mass energy of 8 TeV. 

The final result compares normalized \phistar differential cross-sections of data that was collected at the the Compact Muon Solenoid to multiple different simulations. One of the simulations was used multiple times with different settings in an attempt to match the data more effectively.




%%%%%%%%%%%%%%%%%%%%%%%%%%%%%%%%%%%%%%%%%%%%%%%%%%%%%%%%%%%%%%%%%%%%%%%%%%%%%%%%
