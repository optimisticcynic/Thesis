%%%%%%%%%%%%%%%%%%%%%%%%%%%%%%%%%%%%%%%%%%%%%%%%%%%%%%%%%%%%%%%%%%%%%%%%%%%%%%%%
% conclusion.tex:
%%%%%%%%%%%%%%%%%%%%%%%%%%%%%%%%%%%%%%%%%%%%%%%%%%%%%%%%%%%%%%%%%%%%%%%%%%%%%%%%
\chapter{Conclusion and Discussion}
The measured, normalized cross-section of the variable \phistar of the \Z boson decay to electron pairs was preformed using 19.7 fb$^{-1}$ of $\sqrt(s)=\SI{8}{TeV}$ data collected with the CMS detector. This was then unfolded using a \MADGRAGH sample and compared \POWHEG + \PYTHIAeight sample, using different ``tunes".

All \POWHEG + \PYTHIAeight tunes match unfolded data within 5\%  for $\phistar<0.2$ and are practically identical for higher \phistar. For $\phistar<0.03$ smaller values of either $\SigmaHard$ or $\SigmaSoft$ decreased the disagreement between data and theory. This was expected since lowering either variable will result in a smaller \bosonpt of the \Z and reduce the deficit of low \bosonpt \Z bosons. However, major changes of the tune, of the order of decreasing both $\SigmaHard$ and $\SigmaSoft$ by a factor of two, were unable to completely remove the disagreement. It therefore does not appear to be possible to remove the disagreement at low \phistar of the theory with data by making a  reasonable change the either $\SigmaHard$ or $\SigmaSoft$. It also appeared that for medium \phistar region($0.03<\phistar<0.15$) lowering $\SigmaHard$ or $\SigmaSoft$  increased the disagreement. For high \phistar the tune has a trivial effect with all tunes overestimating the cross-section by the same amount within uncertainty. Therefore it does not appear that the variables $\SigmaHard$ or $\SigmaSoft$ are responsible for the disagreement between data and theory. There are however many other free parameters used by \PYTHIAeight that could possibly be partially responsible for the disagreements between data and theory, such as the PDF used which could be studied by another analysis in the future. The ability to create settings in the hadronizer that allow for a more accurate simulation would be useful for many situations, such as was mentioned earlier, measuring the mass of the W boson.  A second important reason is that if the hadronizer is not responsible for the difference between theory and measured results. This may imply a problem with the standard model. Therefore, in the future it is important that  studies attempt to find the reason for the difference between theory and data, and if possible fix it. 
\label{conclusion_chapter}
%%%%%%%%%%%%%%%%%%%%%%%%%%%%%%%%%%%%%%%%%%%%%%%%%%%%%%%%%%%%%%%%%%%%%%%%%%%%%%%%

%%%%%%%%%%%%%%%%%%%%%%%%%%%%%%%%%%%%%%%%%%%%%%%%%%%%%%%%%%%%%%%%%%%%%%%%%%%%%%%%
